\documentclass[a4paper,12pt]{article} % Prepara un documento per carta A4, con un bel font grande

\usepackage[italian]{babel} % Adatta LaTeX alle convenzioni tipografiche italiane,
% e ridefinisce alcuni titoli in italiano, come "Capitolo" al posto di "Chapter",
% se il vostro documento è in italiano
\usepackage[T1]{fontenc} % Riga da togliere se si compila con PDFLaTeX
\usepackage[utf8]{inputenc} % Consente l'uso caratteri accentati italiani

\frenchspacing % forza LaTeX ad una spaziatura uniforme, invece di lasciare più spazio
% alla fine dei punti fermi come da convenzione inglese

\title{Esempio di documento in \LaTeX} % \LaTeX è una macro che compone il logo "LaTeX"
% I commenti (introdotti da %) vengono ignorati

\author{Mario Rossi}
\date{8 aprile 2002}
% in alternativa a \date il comando \today introduce la data di sistema.

\begin{document}
\maketitle % Produce il titolo a partire dai comandi \title, \author e \date

\begin{abstract} % Questo è l'inizio dell'ambiente "abstract".
% L'ambiente abstract è fatto per contenere un riassunto del contenuto.
Breve dimostrazione dell'uso di \LaTeX.
\end{abstract} % Qui termina l'ambiente ''abstract''

\tableofcontents % Prepara l'indice generale

\section{Testo normale} %
È possibile scrivere il testo dell'articolo normalmente, ed 
\emph{enfatizzare} alcune parti del discorso. %
Una riga vuota nel testo indica la fine di un paragrafo.

Quindi questo è un nuovo paragrafo.

\section{Formule} %
La forza di \LaTeX\ sono però le formule, sia in linea (ad esempio \(y=x^2\))
 che messe in bella mostra in un'area propria:
\[y=\sqrt{x+y}\]

\section{Poesia} %
L'ambiente ``verse'' è usato per comporre tipograficamente le poesie:
\begin{verse}
La vispa Teresa avea tra l'erbetta\\ % la doppia barra inversa forza a capo
al volo sorpresa gentil farfalletta.
\end{verse}
\end{document}